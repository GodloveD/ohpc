\newcolumntype{C}[1]{>{\centering}p{#1}} 
\newcolumntype{L}[1]{>{\raggedleft}p{#1}} 
\small
\begin{tabularx}{\textwidth}{L{\firstColWidth{}}|C{\secondColWidth{}}|X}
\toprule
{\bf RPM Package Name} & {\bf Version} & {\bf Info/URL}  \\ 
\midrule

pdtoolkit-gnu12-ohpc &
\multirow{2}{*}{3.25.1} & 
\multirow{2}{\linewidth}{PDT is a framework for analyzing source code. \newline {\color{logoblue} \url{http://www.cs.uoregon.edu/Research/pdt}}} \\ 
pdtoolkit-intel-ohpc &
& \\ 
\hline
% <-- end entry for pdtoolkit

% <-- begin entry for scalasca
scalasca-gnu12-impi-ohpc &
\multirow{8}{*}{2.5} & 
\multirow{8}{\linewidth}{Toolset for performance analysis of large-scale parallel applications. \newline {\color{logoblue} \url{http://www.scalasca.org}}} \\ 
scalasca-gnu12-mpich-ohpc &
& \\ 
scalasca-gnu12-mvapich2-ohpc &
& \\ 
scalasca-gnu12-openmpi4-ohpc &
& \\ 
scalasca-intel-impi-ohpc &
& \\ 
scalasca-intel-mpich-ohpc &
& \\ 
scalasca-intel-mvapich2-ohpc &
& \\ 
scalasca-intel-openmpi4-ohpc &
& \\ 
\hline
% <-- end entry for scalasca

% <-- begin entry for scorep
scorep-gnu12-impi-ohpc &
\multirow{8}{*}{7.1} & 
\multirow{8}{\linewidth}{Scalable Performance Measurement Infrastructure for Parallel Codes. \newline {\color{logoblue} \url{http://www.vi-hps.org/projects/score-p}}} \\ 
scorep-gnu12-mpich-ohpc &
& \\ 
scorep-gnu12-mvapich2-ohpc &
& \\ 
scorep-gnu12-openmpi4-ohpc &
& \\ 
scorep-intel-impi-ohpc &
& \\ 
scorep-intel-mpich-ohpc &
& \\ 
scorep-intel-mvapich2-ohpc &
& \\ 
scorep-intel-openmpi4-ohpc &
& \\ 
\hline
% <-- end entry for scorep

% <-- begin entry for tau
tau-gnu12-impi-ohpc &
\multirow{8}{*}{2.31.1} & 
\multirow{8}{\linewidth}{Tuning and Analysis Utilities Profiling Package. \newline {\color{logoblue} \url{http://www.cs.uoregon.edu/research/tau/home.php}}} \\ 
tau-gnu12-mpich-ohpc &
& \\ 
tau-gnu12-mvapich2-ohpc &
& \\ 
tau-gnu12-openmpi4-ohpc &
& \\ 
tau-intel-impi-ohpc &
& \\ 
tau-intel-mpich-ohpc &
& \\ 
tau-intel-mvapich2-ohpc &
& \\ 
tau-intel-openmpi4-ohpc &
& \\ 
\hline
% <-- end entry for tau

\bottomrule
\end{tabularx}
