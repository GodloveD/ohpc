\clearpage

\definecolor{Gray}{gray}{0.5}
\newcommand{\captionSpace}{-0.15cm}
\newcommand{\tabSpaceBot}{1.0cm}
\captionsetup{justification=raggedright,singlelinecheck=false}

\subsection{Package Manifest} \label {appendix:manifest}

\vspace*{0.25cm}
This appendix provides a summary of available meta-package groupings and all of
the individual RPM packages that are available as part of this \OHPC{}
release. The meta-packages provide a mechanism to group related collections of
RPMs by functionality and provide a convenience mechanism for installation.  A
list of the available meta-packages and a brief description is presented in
Table~\ref{table:groups}.

\vspace*{1.25cm}
\begin{table}[h!] 
\caption{\bf Available \OHPC{} Meta-packages} \vspace*{\captionSpace{}}
\label{table:groups}
\input manifest/patterns
\end{table}

% meta-packages (2)
\begin{table}[h!] 
\caption*{Table~\ref{table:groups} (cont): {\bf Available \OHPC{} Meta-packages} \vspace*{\captionSpace{}} }
\input manifest/patterns2
\end{table}

\iftoggleverb{isx86}
% meta-packages (3)
\begin{table}[h!] 
\caption*{Table~\ref{table:groups} (cont): {\bf Available \OHPC{} Meta-packages} \vspace*{\captionSpace{}} }
\input manifest/patterns3
\end{table}

\fi

\clearpage
What follows next in this Appendix is a series of tables that summarize the
underlying RPM packages available in this \OHPC{} release. These packages are
organized by groupings based on their general functionality and each table
provides information for the specific RPM name, version, brief summary, and the
web URL where additional information can be obtained for the component. Note
that many of the 3rd party community libraries that are pre-packaged
with \OHPC{} are built using multiple compiler and MPI families. In these cases,
the RPM package name includes delimiters identifying the development
environment for which each package build is targeted.  Additional information
on the \OHPC{} package naming scheme is presented in \S\ref{sec:3rdparty}. 
The relevant package groupings and associated Table references are as follows:

\vspace*{0.1cm}

\begin{itemize*}
\item Administrative tools (Table~\ref{table:admin})
\iftoggleverb{isWarewulf}
\item Provisioning (Table~\ref{table:provisioning})
\fi
\item Resource management (Table~\ref{table:rms})
\item Compiler families (Table~\ref{table:compiler-families})
\item MPI families (Table~\ref{table:mpi-families})
\item Development tools (Table~\ref{table:dev-tools})
\item Performance analysis tools (Table~\ref{table:perf-tools})
\iftoggleverb{isCentOS_x86}
\item Lustre (Table~\ref{table:lustre})
\fi

%\item Distro support packages and dependencies (Table~\ref{table:distro-packages})
\item IO Libraries (Table~\ref{table:io-libs})
\item Runtimes (Table~\ref{table:runtimes})
\item Serial/Threaded Libraries (Table~\ref{table:serial-libs})
\item Parallel Libraries (Table~\ref{table:parallel-libs})
\end{itemize*}
