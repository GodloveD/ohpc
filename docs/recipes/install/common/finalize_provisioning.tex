\subsection{Finalizing provisioning configuration} \label{sec:assemble_bootstrap}

\Warewulf{} employs a two-stage boot process for provisioning nodes via
creation of a bootstrap image that is used to initialize the process, and a virtual node
file system capsule containing the full system image. This section highlights
creation of the necessary provisioning images, followed by the registration of
desired compute nodes.

\subsubsection{Assemble bootstrap image}

The bootstrap image includes the runtime kernel and associated modules, as well
as some simple scripts to complete the provisioning process.

%\iftoggle{isCentOS_ww_slurm_aarch}{\clearpage}

% begin_ohpc_run
% ohpc_comment_header Assemble bootstrap image \ref{sec:assemble_bootstrap}
\begin{lstlisting}[language=bash,literate={-}{-}1,keywords={},upquote=true]
# Build bootstrap image
[sms](*\#*) wwctl container build rocky-8.9
\end{lstlisting}
% end_ohpc_run

\iftoggle{isCentOS_ww_slurm_aarch}{\vspace*{0.4cm}}

\iftoggle{isSLES_ww_slurm_aarch}{\vspace*{-0.1cm}}

\subsubsection{Register nodes for provisioning}

Nodes can be registered for provisioning using the following syntax.
