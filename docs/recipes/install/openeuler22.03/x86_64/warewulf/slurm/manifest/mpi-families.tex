\newcolumntype{C}[1]{>{\centering}p{#1}} 
\newcolumntype{L}[1]{>{\raggedleft}p{#1}} 
\newcolumntype{C}[1]{>{\centering}p{#1}} 
\newcolumntype{L}[1]{>{\raggedleft}p{#1}} 
\small
\begin{tabularx}{\textwidth}{L{\firstColWidth{}}|C{\secondColWidth{}}|X}
\toprule
{\bf RPM Package Name} & {\bf Version} & {\bf Info/URL}  \\ 
\midrule

% <-- begin entry for intel-psxe-mpi-devel-ohpc
\multirow{2}{*}{intel-psxe-mpi-devel-ohpc} & 
\multirow{2}{*}{2021.1} & 
OpenHPC compatibility package for Intel(R) MPI Library. \newline { \color{logoblue} \url{https://github.com/openhpc/ohpc}} 
\\ \hline 
% <-- end entry for intel-psxe-mpi-devel-ohpc

% <-- begin entry for intel-mpi-devel-ohpc
\multirow{2}{*}{intel-mpi-devel-ohpc} & 
\multirow{2}{*}{2021.1} & 
OpenHPC compatibility package for Intel(R) oneAPI MPI Library.  { \color{logoblue} \url{https://github.com/openhpc/ohpc}} 
\\ \hline 
% <-- end entry for intel-mpi-devel-ohpc

% <-- begin entry for libfabric-ohpc
\multirow{2}{*}{libfabric-ohpc} & 
\multirow{2}{*}{1.13.0} & 
User-space RDMA Fabric Interfaces. \newline { \color{logoblue} \url{http://www.github.com/ofiwg/libfabric}} 
\\ \hline 
% <-- end entry for libfabric-ohpc

% <-- begin entry for mpich-ofi
\multirow{2}{*}{mpich-ofi-gnu9-ohpc} &
\multirow{2}{*}{3.4.2} & 
MPICH MPI implementation. \newline {\color{logoblue} \url{http://www.mpich.org}} \\ 
\hline
% <-- end entry for mpich-ofi

% <-- begin entry for mpich-ucx
mpich-ucx-gnu9-ohpc &
\multirow{2}{*}{3.4.2} & 
\multirow{2}{\linewidth}{MPICH MPI implementation. \newline {\color{logoblue} \url{http://www.mpich.org}}} \\ 
mpich-ucx-intel-ohpc &
& \\ 
\hline
% <-- end entry for mpich-ucx

% <-- begin entry for mpich-ofi
mpich-ofi-gnu12-ohpc &
\multirow{2}{*}{3.4.3} & 
\multirow{2}{\linewidth}{MPICH MPI implementation. \newline {\color{logoblue} \url{http://www.mpich.org}}} \\ 
mpich-ofi-intel-ohpc &
& \\ 
\hline
% <-- end entry for mpich-ofi

% <-- begin entry for mpich-ucx
\multirow{2}{*}{mpich-ucx-gnu12-ohpc} &
\multirow{2}{*}{3.4.3} & 
MPICH MPI implementation. \newline {\color{logoblue} \url{http://www.mpich.org}} \\ 
\hline
% <-- end entry for mpich-ucx

% <-- begin entry for mvapich2-psm2
\multirow{2}{*}{mvapich2-psm2-gnu9-ohpc} &
\multirow{2}{*}{2.3.4} & 
OSU MVAPICH2 MPI implementation. \newline {\color{logoblue} \url{http://mvapich.cse.ohio-state.edu}} \\ 
\hline
% <-- end entry for mvapich2-psm2

% <-- begin entry for mvapich2
mvapich2-gnu9-ohpc &
\multirow{1}{*}{2.3.6} & 
\multirow{5}{\linewidth}{OSU MVAPICH2 MPI implementation. \newline {\color{logoblue} \url{http://mvapich.cse.ohio-state.edu}}} \\ 
\cline{1-2} mvapich2-gnu12-ohpc &\multirow{4}{*}{2.3.7}
& \\ 
mvapich2-intel-ohpc &
& \\ 
mvapich2-psm2-gnu12-ohpc &
& \\ 
mvapich2-psm2-intel-ohpc &
& \\ 
\hline
% <-- end entry for mvapich2

% <-- begin entry for openmpi4
openmpi4-gnu9-ohpc &
\multirow{1}{*}{4.1.1} & 
\multirow{3}{\linewidth}{A powerful implementation of MPI/SHMEM. \newline {\color{logoblue} \url{http://www.open-mpi.org}}} \\ 
\cline{1-2} openmpi4-gnu12-ohpc &\multirow{2}{*}{4.1.4}
& \\ 
openmpi4-intel-ohpc &
& \\ 
\hline
% <-- end entry for openmpi4

% <-- begin entry for ucx-ohpc
\multirow{2}{*}{ucx-ohpc} & 
\multirow{2}{*}{1.11.2} & 
UCX is a communication library implementing high-performance messaging.  { \color{logoblue} \url{http://www.openucx.org}} 
\\ \hline 
% <-- end entry for ucx-ohpc

\bottomrule
\end{tabularx}
